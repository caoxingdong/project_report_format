\documentclass{article}
\usepackage[utf8]{inputenc}
\usepackage{geometry}
\geometry{a4paper,scale=0.8}

\usepackage{listings}

\usepackage{indentfirst}

\renewcommand\thesection{\Roman{section}}
\renewcommand\thesubsection{\Alph{subsection}}

\usepackage{titlesec}
\titlelabel{\thetitle.\quad}
\titleformat*{\section}{\centering}
\titleformat*{\subsection}{\itshape}

\usepackage{times}

\linespread{1.5}

\usepackage{fancyhdr} 
\pagestyle{fancy}
\fancyhf{}
\fancyheadoffset{0cm}
\renewcommand{\headrulewidth}{0pt} 
\renewcommand{\footrulewidth}{0pt}
\fancyhead[R]{\thepage}
\fancypagestyle{plain}{%
	\fancyhf{}%
	\fancyhead[R]{\thepage}%
}


\usepackage{etoolbox}
\patchcmd{\thebibliography}{\section*{\refname}}{}{}{}

% Default fixed font does not support bold face
\DeclareFixedFont{\ttb}{T1}{txtt}{bx}{n}{12} % for bold
\DeclareFixedFont{\ttm}{T1}{txtt}{m}{n}{12}  % for normal

% Custom colors
\usepackage{color}
\definecolor{deepblue}{rgb}{0,0,0.5}
\definecolor{deepred}{rgb}{0.6,0,0}
\definecolor{deepgreen}{rgb}{0,0.5,0}

\usepackage{listings}

% Python style for highlighting
\newcommand\pythonstyle{\lstset{
		language=Python,
		basicstyle=\ttm,
		otherkeywords={self},             % Add keywords here
		keywordstyle=\ttb\color{deepblue},
		emph={MyClass,__init__},          % Custom highlighting
		emphstyle=\ttb\color{deepred},    % Custom highlighting style
		stringstyle=\color{deepgreen},
		frame=tb,                         % Any extra options here
		showstringspaces=false            % 
}}


% Python environment
\lstnewenvironment{python}[1][]
{
	\pythonstyle
	\lstset{#1}
}
{}

% Python for external files
\newcommand\pythonexternal[2][]{{
		\pythonstyle
		\lstinputlisting[#1]{#2}}}

% Python for inline
\newcommand\pythoninline[1]{{\pythonstyle\lstinline!#1!}}


\begin{document}
	\title{Human Activity Action Recognition Approach}
	\author{Leonardo Queiroz, Xingdong Cao}
	\date{}
	\maketitle
	
	\section{\textsc{Introduction and objectives}}
		One of the many applications for a smart system is the ability to provide an automated assessment of health. In the current aging population, “the challenges of maintaining mobility and cognitive function make it increasingly difficult to remain living alone therefore forcing many people to seek residence in clinical institutions” \cite{arcelus2007integration}.
		
	\section{\textsc{Framework}}
		This section describes the proposed approach of using only skeleton points in recognizing a human activity. Fig. 1 illustrates the overall design of the recognition system. The system consists of 4 main components: database, skeleton parser, classifier, and output.
		\subsection{Databases}
		\subsection{Skeleton Parser}
		\subsection{Classifier}
		\subsection{Permormance Metric}
	\section{\textsc{Experimental results}}
	
	\section{\textsc{Future Work}}
	
	\section{\textsc{Conclusions}}
	
	\section*{\textsc{References}}
	    \bibliography{bibtex}{}	    
		\bibliographystyle{plain}
		
	\section*{\textsc{Appendix}} 	
	\pythonexternal{pythoncode.py}
	
\end{document}